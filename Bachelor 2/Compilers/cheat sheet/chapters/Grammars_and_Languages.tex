\subsection{Grammars}

\subsubsection{LL(k)}
1. Identify Production Rules for Each Non-Terminal

    Initialization:
    For each non-terminal \( A \), list all production rules of the form \( A \rightarrow \alpha_1 \) and \( A \rightarrow \alpha_2 \), where \( \alpha_1 \) and \( \alpha_2 \) are sequences of symbols (terminals and/or non-terminals).
    Compare Productions:
    For each pair of distinct production rules \( A \rightarrow \alpha_1 \) and \( A \rightarrow \alpha_2 \):
        Note that the possible right-hand sides are \( \alpha_1 \) and \( \alpha_2 \).

2. Analyze Derivations with Lookahead

    Derive Sentential Forms:
    For each pair \( (A \rightarrow \alpha_1, A \rightarrow \alpha_2) \), consider the derivations of the form:
        \( S \Rightarrow^* w A \gamma \Rightarrow w \alpha_1 \gamma \Rightarrow^* w x_1 \)
        \( S \Rightarrow^* w A \gamma \Rightarrow w \alpha_2 \gamma \Rightarrow^* w x_2 \)
    Here, \( w \) is a prefix of terminals, and \( \gamma \) is a sequence of symbols.
    Compute Lookahead Sets:
    Determine the lookahead sets \( \text{First}_k(x_1) \) and \( \text{First}_k(x_2) \) where \( x_1 \) and \( x_2 \) are terminal strings derived from \( \alpha_1 \gamma \) and \( \alpha_2 \gamma \), respectively.

3. Check for Disambiguation

    Lookahead Comparison:
    For each pair \( (A \rightarrow \alpha_1, A \rightarrow \alpha_2) \), compare the lookahead sets:
        Check if \( \text{First}_k(x_1) \) equals \( \text{First}_k(x_2) \).
    Verify Consistency:
    If \( \text{First}_k(x_1) = \text{First}_k(x_2) \), ensure that \( \alpha_1 \) and \( \alpha_2 \) are identical:
        If \( \alpha_1 \neq \alpha_2 \) when the lookahead sets are equal, the grammar is not LL(k).

4. Conclude LL(k) Status

    If all pairs of distinct production rules for every non-terminal either have different lookahead sets or identical right-hand sides, the grammar is LL(k).
    If any pair of rules has the same lookahead sets but different right-hand sides, the grammar is not LL(k).


\subsection{Strong LL(k)}    
A $CFG G= \langle V ,T ,P,S \rangle $ is strong LL(k) 
iff, for all pairs of rules $A→ \alpha_1$ and $A→ \alpha_2$ in $P ($with $\alpha_1 = \alpha_2)$:
$First^k ( \alpha_1 Follow^k (A) ) \cap First^k (\alpha_2 Followk (A) ) = \varnothing $

%SLR(k)

%LALR(k)

\subsubsection{LR(k)}
The main limitation to the bottom-up
parser we have just defined is that it is non-deterministic. There are two
sources of non-determinism in this parser:

Reduce-Reduce conflict: such conflicts occur when the top of the stack is
the handle to two different rules, and the parser cannot decide which
rule to reduce;

Shift-Reduce conflicts: such conflicts occur when the top of the stack con-
stitutes a handle to some rule, but the parser cannot determine whether
it should continue shifting or whether it should reduce now.




%To determine if a grammar is LR(1), you need to check if it can be parsed using an LR(1) parser without conflicts. An LR(1) parser is a type of bottom-up parser that reads input from left to right and produces a rightmost derivation in reverse. The "1" in LR(1) indicates that the parser uses 1 lookahead token to make parsing decisions.

%Here's a step-by-step method to determine if a grammar is LR(1):

%\textbf{1. Construct the LR(1) Items}\\

%LR(1) items are similar to LR(0) items but with an additional lookahead component. An LR(1) item is of the form `[A → α $\bullet$ β, a]` where:

%`A → αβ` is a production.
%`$\bullet$` indicates the current position in the production.
%`a` is a lookahead token (a terminal that may appear next in the input).

%To construct the LR(1) items:

%Start with an augmented grammar by adding a new start symbol \(S'\) with a production \(S' \to S\), where \(S\) is the original start symbol of the grammar.

%Create the initial LR(1) item for the augmented production: `[(S' → $\bullet$S, $)]`, where `$` is the end-of-input marker.

%\textbf{2. Compute the Closure of LR(1) Items}\\

%The closure of a set of LR(1) items includes all items that could be valid in that parsing context. To compute the closure:

%1. Start with an initial set of items (e.g., the initial LR(1) item).
%2. For each item `[A → α $\bullet$ Bβ, a]` in the set, if `$\bullet$` is before a non-terminal `B`, for every production `B → γ`:
%   For each terminal `b` in FIRST(βa), add the item `[B → $\bullet$γ, b]` to the set if it is not already present.
%3. Repeat until no more items can be added.

%\textbf{3. Construct the Canonical Collection of LR(1) Sets}\\

%The canonical collection is a set of all unique LR(1) item sets (states). To construct it:

%1. Start with the closure of the initial LR(1) item set as the first state.
%2. For each state and each grammar symbol (terminal or non-terminal), compute the goto for that symbol:
%   The goto of a set of items for a symbol `X` is a set of items you get by moving the `$\bullet$` past `X` in all items and then taking the closure of the resulting set.
%3. If the resulting set of items is new (i.e., not already in the collection), add it as a new state.
%4. Repeat until no new states can be added.

%\textbf{4. Construct the LR(1) Parsing Table}\\

%Using the canonical collection of LR(1) sets, construct the parsing table:

%Action Table: Specifies parser actions (shift, reduce, accept) for terminals.
%Goto Table: Specifies transitions between states for non-terminals.

%To fill in the table:

%Shift: For each item `[A → α $\bullet$ aβ, b]` where `a` is a terminal, if `goto(I, a) = J`, then set `ACTION[I, a]` to "shift J".
%Reduce: For each item `[A → α $\bullet$, a]`, set `ACTION[I, a]` to "reduce A → α".
%Accept: If the item is `[S' → S$\bullet$, $]` in some state, set `ACTION[I, $]` to "accept".
%Goto: If `goto(I, A) = J` for non-terminal `A`, set `GOTO[I, A] = J`.

%\textbf{5. Check for Conflicts}\\

%An LR(1) grammar must have no parsing conflicts:

%Shift-Reduce Conflict: Occurs when a table entry requires both a shift and a reduce action for the same terminal.
%Reduce-Reduce Conflict: Occurs when a table entry requires more than one reduce action for the same terminal.

%\textbf{6. Determine if the Grammar is LR(1)}\\

%If the LR(1) parsing table has no conflicts (no shift-reduce or reduce-reduce conflicts), then the grammar is LR(1).


%\subsection{Languages}

%\subsubsection{Regular langauges}
%Regular languages are those that can be recognized by a finite automaton or expressed by a regular expression.

%\subsubsection{Context-free languages}
%Context-free languages are those that can be recognized by a pushdown automaton or expressed by a context-free grammar.
