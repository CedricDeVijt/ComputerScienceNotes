\subsection{Grammars}

\subsubsection{LL(k)}

1. No Left Recursion: No non-terminal should be able to derive itself through a sequence of productions that starts with the same non-terminal. For example, a rule like $A \rightarrow A\alpha$ (where $\alpha$  is any sequence of terminals and/or non-terminals) would be left-recursive.

2. Left Factoring: For any non-terminal $A$ and productions $A \rightarrow \alpha \beta_1$ and $A \rightarrow \alpha \beta_2$ where $\alpha$ is a common prefix, the grammar should be refactored to remove the common prefix. The left-factored form would be:
   $A \rightarrow \alpha A'$\\
   $A' \rightarrow \beta_1 \mid \beta_2$

3. First/Follow Set Conditions: For each non-terminal $A$ and productions $A \rightarrow \alpha_1$ and $A \rightarrow \alpha_2$:
   The sets of terminals that can appear as the first token of the strings derived from $\alpha_1$ and $\alpha_2$ (i.e., the FIRST sets) must be disjoint.
   If $\alpha_i$ can derive the empty string $\epsilon$, then the FIRST set of $\alpha_i$ should not intersect with the FOLLOW set of $A$. The FOLLOW set of $A$ is the set of terminals that can appear immediately to the right of $A$ in some sentential form.



%SLR(k)

%LALR(k)

%LR(k)

\subsection{Languages}

\subsubsection{Regular langauges}
Regular languages are those that can be recognized by a finite automaton or expressed by a regular expression.

\subsubsection{Context-free languages}
Context-free languages are those that can be recognized by a pushdown automaton or expressed by a context-free grammar.