%DONE

\section{Example Languages}
\textbf{LR(0) language that is not LL(1)}\\
    $L = \{a^n b^n | n \geq 0\}$

\textbf{LR(0) language that is not LL(k) for any k}\\
$L = \{a^n b^n c^m | n, m \geq 1\} $\

\textbf{LL(2) language that is not LL(1)}\\
$L = \{ a^n b c^n d \mid n \geq 1 \} \cup \{ a^n c b^n d \mid n \geq 1 \}$

\textbf{CFL that is not LR(1)}\\
    English language

\subsection{Example Grammars}
\subsubsection{LL(k) Grammar}
\textbf{LL(1) grammar that is not strongly LL(1)}\\ % Cursus p146 Theorem 5.4.
    Theorem 5.4 "All LL(1) grammars are also strong LL(1), i.e. the classes of LL(1) and strong LL(1) grammars coincide."
    
\textbf{LL(1) grammar that is not LALR(1)}\\ % Cursus p205 Figure 6.23: An grammar which is LL(1) and not LALR(1)
    $S\rightarrow aX$\\
    $S\rightarrow Eb$\\
    $S\rightarrow F c$\\
    $X\rightarrow Ec$\\
    $X\rightarrow F b$\\
    $E\rightarrow A$\\
    $F\rightarrow A$\\
    $A\rightarrow \epsilon$\\

\textbf{LL(2) grammar that is not strongly LL(2)}\\ % Cursus p144 Example 5.11 + p146: "Example 5.11, which is LL(2), and let us show that it is not strong LL(2)"
    $S \rightarrow aAa$\\
    $S \rightarrow bABa$\\
    $A \rightarrow b$\\
    $A \rightarrow \epsilon$\\
    $B \rightarrow b$\\
    $B \rightarrow c$\\
    
\textbf{LL(2) and SLR(1) grammar but neither LL(1) nor LR(0)} % Cursus p205 Figure 6.24: A grammar that is LL(2) and SLR(1) but neither LL(1) nor LR(0).
    $S\rightarrow ab$\\
    $S\rightarrow ac$\\
    $S\rightarrow a$\\
    
\subsubsection{LALR(k) Grammar}
\textbf{LALR(1) grammar that is not SLR(1)}\\ % Cursus p197 Figure 6.17: An LALR(1) grammar that is not SLR(1) and not LL(1) but LL(2).
    $S \rightarrow Aa$\\
    $S \rightarrow bAc$\\
    $S \rightarrow dc$\\
    $A \rightarrow d$\\
    
\subsubsection{LR(k) Grammar}
\textbf{LR(1) grammar that is not LALR(1) and not LL(1)}\\ % Cursus p198 Figure 6.18: An LR(1) grammar that is not LALR(1) and not LL(1).
    $S\rightarrow aAd$\\
    $S\rightarrow bBd$\\
    $S\rightarrow aBe$\\
    $S\rightarrow bAe$\\
    $A\rightarrow c$\\
    $B\rightarrow c$\\

\textbf{LR(k + 1) grammar that is not LR(k) and not LL(k + 1)}\\ % Cursus p198 Figure 6.19: An LR(k + 1) grammar that is not LR(k) and not LL(k + 1).
    $S\rightarrow Ab^k c$\\
    $S\rightarrow Bb^k d$\\
    $A\rightarrow a$\\
    $B\rightarrow a$\\

\subsubsection{Context Free Grammar}
\textbf{CFG that is not LR(k)}\\ % Cursus p199 Figure 6.20: A grammar which is not LR(k) for any k.
    $S \rightarrow aAc$\\
    $A \rightarrow bAb$\\
    $A  \rightarrow b$\\
    
\textbf{CFG that is not LR(0)}\\ % Cursus p190 Figure 6.14: A simple grammar which is not LR(0).
    $S \rightarrow a$\\
    $S \rightarrow abS$\\
    

\subsubsection{SLR(k) Grammar}
\textbf{SLR(1) grammar that is not LR(0)}\\ % Cursus 176 + p197 "Figure 6.7 is SLR(1) but not LR(0)"
    $S\rightarrow Exp$\\
    $Exp\rightarrow Exp + Prod$\\
    $Exp\rightarrow Prod$\\
    $Prod\rightarrow Prod * Atom$\\
    $Prod\rightarrow Atom$\\
    $Atom\rightarrow Id$\\
    $Atom\rightarrow (Exp)$\\

\subsection{Special examples}
\textbf{LL(1) language with a non-LL(1) grammar}
    $L = \{ w \mid w \text{ is a string of balanced parentheses} \}$\\
Grammar:\\
    $S \rightarrow S S$\\
    $S \rightarrow ( S )$\\
    $S \rightarrow \epsilon$\\

