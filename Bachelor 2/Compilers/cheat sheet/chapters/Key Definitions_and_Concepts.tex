% DONE

\section{Key Definitions and Concepts}
\textbf{Regular Language} A language that can be expressed using regular expressions or recognized by finite automata (DFA or NFA). Regular languages are closed under union, concatenation, and Kleene star operations.  Regular languages are less expressive than context-free languages.
\textbf{Context-Free Grammar (CFG)} A grammar consisting of rules of the form $A \rightarrow \alpha $, where $A$ is a nonterminal, and $\alpha $ is a string of terminals and\/or nonterminals. CFGs generate context-free languages, which can be recognized by pushdown automata.
\textbf{Context-Free Language (CFL)}: if it can be generated by a context-free grammar (CFG)
\textbf{Finite-State Machine (FSM)} A model of computation with states, transitions, and an alphabet. A deterministic finite automaton (DFA) has exactly one transition per input symbol per state, while a nondeterministic finite automaton (NFA) may have multiple or $\epsilon$-transitions.
\textbf{Pushdown Automaton (PDA)} An automaton with a stack, capable of recognizing context-free languages. A deterministic PDA (DPDA) has at most one possible move per configuration, while a nondeterministic PDA (NPDA) may have multiple.
\textbf{LL(k) Grammar} A grammar that can be parsed by a top-down parser with k-token lookahead, where the parser can deterministically choose the next production based on the first k tokens.
\textbf{LR(k) Grammar} A grammar that can be parsed by a bottom-up parser with k-token lookahead, using a shift-reduce strategy. Variants include SLR(1) (simple LR), LALR(1) (lookahead LR), and LR(1) (canonical LR).
\textbf{First(k) Set} For a nonterminal $A$, the set of all possible $k$-token prefixes that can begin a string derived from $A$.
\textbf{Follow(k) Set} For a nonterminal $A$, the set of all possible $k$-token prefixes that can appear immediately after $A$ in a derivation.
\textbf{Closure in Parsing} In LR parsing, the closure of a set of items (e.g., $A \rightarrow \alpha \cdot \beta $) includes all items that can be reached by following nonterminal transitions (adding rules like $B \rightarrow \cdot \gamma$ if $\beta $ starts with $B$).
\textbf{Regular Expression} A pattern describing a regular language, using symbols like `*` (zero or more), `+` (one or more), `$\vert$` (union), and `()` (grouping).
\textbf{Longest Match Principle (Maximal munch)} In lexical analysis, the scanner selects the longest prefix of the input that matches a token pattern.
\textbf{Strong Language} A context-free language is strong if it can be parsed deterministically by an LR(1) parser. Strong languages are a proper subset of context-free languages and include all LR(k) languages. They are characterized by having unambiguous grammars that can be parsed bottom-up with bounded lookahead.
\textbf{Strong Grammar} A context-free grammar is strong if it generates a strong language and can be parsed deterministically by an LR(1) parser without conflicts. Strong grammars are unambiguous and have the property that every viable prefix can be extended to a complete derivation in at most one way with bounded lookahead.
