\section{Parsing Grammar Proofs}

\subsection{LL(k) Proof}
A grammar is \textbf{LL(k)} if:
\begin{enumerate}
  \item No left recursion exists.
  \item For every non-terminal \(A\) with productions \(A \to \alpha \mid \beta\):
  \[
  \text{FIRST}_k(\alpha \cdot \text{FOLLOW}_k(A)) \cap \text{FIRST}_k(\beta \cdot \text{FOLLOW}_k(A)) = \emptyset
  \]
  (If \(\alpha\)/\(\beta\) can derive \(\epsilon\), include \(\text{FOLLOW}_k(A)\).)
\end{enumerate}

\subsection{SLR(k) Proof}
A grammar is \textbf{SLR(k)} if:
\begin{enumerate}
  \item Construct SLR(1) states (LR(0) items + \text{FOLLOW} for reduce actions).
  \item No \textit{shift-reduce} or \textit{reduce-reduce} conflicts in the parsing table.
  \item Conflicts tested using \text{FOLLOW} sets (not lookaheads).
\end{enumerate}

\subsection{LR(k) Proof}
A grammar is \textbf{LR(k)} if:
\begin{enumerate}
  \item Construct canonical LR(k) states (with full lookaheads).
  \item No conflicts in the parsing table (even with precise lookaheads).
\end{enumerate}

\subsection{LALR(k) Proof}
A grammar is \textbf{LALR(k)} if:
\begin{enumerate}
  \item Merge LR(1) states with identical \textit{cores} (same productions, different lookaheads).
  \item Ensure no conflicts arise after merging (common in practice but weaker than LR).
\end{enumerate}

\subsubsection*{Hierarchy}
\[
\text{SLR} \subset \text{LALR} \subset \text{LR}, \quad \text{LL} \subset \text{CFG (non-overlapping)}
\]

\textbf{Key Notes}:
- For \textbf{LL(k)}: Compute \(\text{FIRST}_k\) and \(\text{FOLLOW}_k\) rigorously.
- For \textbf{SLR}: Conflicts = overlap between shift terminals and \(\text{FOLLOW}\) of reduced non-terminal.
- \textbf{LALR} merges LR(1) states; conflicts here imply not LALR but might still be LR.
- \textbf{Left recursion} invalidates LL(k); \textbf{ambiguity} invalidates all.

\subsection{Constructing an LR Parsing Table}
To determine whether a grammar is LR, SLR, or LALR, an LR parsing table must be constructed. The steps are as follows:

\begin{enumerate}
  \item \textbf{Augment the Grammar}: Add a new start symbol \(S'\) and production \(S' \to S\), where \(S\) is the original start symbol. This ensures the parser recognizes when it has fully parsed the input.

  \item \textbf{Compute LR(0) Items}: Create the set of LR(0) items for the grammar. Each item represents a position within a production (e.g., \(A \to \alpha \cdot \beta\)).

  \item \textbf{Closure Operation}: For each item \(A \to \alpha \cdot B \beta\), add all productions of \(B\) with a dot at the beginning (\(B \to \cdot \gamma\)) to the closure. Repeat until no new items can be added.

  \item \textbf{Goto Function}: For each item set \(I\) and each grammar symbol \(X\) (terminal or non-terminal), compute the \textit{goto}(I, X) by advancing the dot over \(X\) in all items of \(I\) and taking the closure of the result.

  \item \textbf{Build the Canonical Collection}: Start with the initial state \(I_0 = \text{closure}(\{S' \to \cdot S\})\). Use the \textit{goto} function to generate all reachable states. This forms the canonical collection of LR(0) item sets.

  \item \textbf{Action and Goto Tables}:
  \begin{itemize}
    \item For each state $I_i$ and terminal $a$, if $\textit{goto}(I_i, a)$ leads to state \(I_j\), add a \textbf{shift} action to the parsing table: \(\text{ACTION}[i, a] = \text{shift } j\).
    \item For each state \(I_i\) containing an item \(A \to \alpha \cdot\) (complete production), add a \textbf{reduce} action: \(\text{ACTION}[i, a] = \text{reduce } A \to \alpha\) for all \(a \in \text{FOLLOW}(A)\) (for SLR). For LR(1) and LALR, use the lookahead symbols associated with the item.
    \item For the state containing \(S' \to S \cdot\), add an \textbf{accept} action: \(\text{ACTION}[i, \$] = \text{accept}\).
    \item For each state \(I_i\) and non-terminal \(A\), if \textit{goto}(I_i, A) leads to state \(I_j\), add \(\text{GOTO}[i, A] = j\).
  \end{itemize}

  \item \textbf{Check for Conflicts}:
  \begin{itemize}
    \item \textbf{Shift-Reduce Conflict}: Occurs if a state has both a shift and reduce action for the same terminal.
    \item \textbf{Reduce-Reduce Conflict}: Occurs if a state has multiple reduce actions for the same terminal.
    \item If no conflicts exist, the grammar is:
    \begin{itemize}
      \item \textbf{SLR}: If conflicts are resolved using \(\text{FOLLOW}\) sets.
      \item \textbf{LR(1)}: If conflicts are resolved using precise lookahead symbols.
      \item \textbf{LALR}: If the grammar remains conflict-free after merging LR(1) states with identical cores.
    \end{itemize}
  \end{itemize}

  \item \textbf{Example Construction}:
  Consider the grammar:
  \[
  S \to E \\
  E \to E + T \mid T \\
  T \to T * F \mid F \\
  F \to (E) \mid id
  \]
  Augment the grammar, compute the LR(0) items, build the canonical collection, and populate the parsing table. Check for conflicts to classify the grammar.

\end{enumerate}

\textbf{Notes on Classification}:
- If the grammar has no conflicts in the LR(0) table, it is \textbf{LR(0)}.
- If conflicts are resolved using \(\text{FOLLOW}\) sets, it is \textbf{SLR}.
- If conflicts are resolved using precise lookaheads, it is \textbf{LR(1)}.
- If merging LR(1) states resolves conflicts, it is \textbf{LALR}.
