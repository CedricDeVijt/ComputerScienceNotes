\section{First and Follow Sets}

The $\mathbf{FIRST^k(\alpha)}$ set for a string $\alpha$ (terminal, non-terminal, or sequence) contains all sequences of up to $k$ terminals that can appear as the first $k$ symbols in any derivation of $\alpha$. If $\alpha$ derives a string shorter than $k$, the entire derived string is included.

Steps to Compute $FIRST^k(\alpha)$:\\
1. If $\alpha$ is a terminal $a$: Add the string $a$ (of length 1) to $FIRST^k(\alpha)$.
2. If $\alpha$ is a sequence $\alpha_1 \alpha_2 \dots \alpha_n$:
   - Initialize $FIRST^k(\alpha) = \emptyset$.
   - For $i = 1$ to $n$:
     - Add to $FIRST^k(\alpha)$ all strings of length up to $k$ from $FIRST^k(\alpha_i)$ if $\alpha_1, \dots, \alpha_{i-1}$ can all derive the empty string $\epsilon$.
     - For strings shorter than $k$, concatenate with $FIRST^{k-|s|}(\alpha_{i+1} \dots \alpha_n)$ (where $|s|$ is the length of the string).
   - If $\alpha$ can derive $\epsilon$, include $\epsilon$ in $FIRST^k(\alpha)$.
3. If $\alpha$ is a non-terminal $A$:
   - For each production $A \to \beta$, compute $FIRST^k(\beta)$ and add its strings to $FIRST^k(A)$.
   - Repeat until $FIRST^k(A)$ stabilizes (no new strings are added).
4. Handle $k$-length lookahead: Truncate strings longer than $k$ to their first $k$ symbols.

The $\mathbf{FOLLOW^k(A)}$ set for a non-terminal $A$ contains all sequences of up to $k$ terminals that can appear immediately after $A$ in some derivation from the start symbol. If $A$ appears at the end of a derivation, include the end-of-input marker (e.g., $\$$).

Steps to Compute $FOLLOW^k(A)$:\\
1. Initialize:
   - For the start symbol $S$, add $\$$ (or a $k$-length end marker) to $FOLLOW^k(S)$.
   - Set $FOLLOW^k(A) = \emptyset$ for all other non-terminals $A$.
2. For each production $B \to \alpha A \beta$ (where $A$ is a non-terminal):
   - Compute $FIRST^k(\beta)$ and add its strings to $FOLLOW^k(A)$.
   - If $\beta$ can derive $\epsilon$, add $FOLLOW^k(B)$ to $FOLLOW^k(A)$.
3. Iterate:
   - Repeat step 2 across all productions until no new strings are added to any $FOLLOW^k(A)$.
4. Handle $k$-length lookahead: Ensure all strings in $FOLLOW^k(A)$ are of length up to $k$, truncating longer strings to their first $k$ symbols.



\section{LR Items}

An $\mathbf{LR(k) \text{ item}}$ is a production with a dot ($\bullet$) marking a position in the right-hand side, along with a $k$-symbol lookahead string. The dot indicates how much of the production has been recognized so far. An item $[A \to \alpha \bullet \beta, w]$ means we have seen $\alpha$ and expect to see $\beta$, with lookahead $w$.

Types of LR Items:
\textbf{Shift item}: $[A \to \alpha \bullet a \beta, w]$ where $a$ is a terminal
\textbf{Reduce item}: $[A \to \alpha \bullet, w]$ where the dot is at the end
\textbf{Goto item}: $[A \to \alpha \bullet B \beta, w]$ where $B$ is a non-terminal

Steps to Compute $\mathbf{CLOSURE(I)}$ for a set of items $I$:
1. Initialize $CLOSURE(I) = I$.
2. For each item $[A \to \alpha \bullet B \beta, w]$ in $CLOSURE(I)$ where $B$ is a non-terminal:
   - For each production $B \to \gamma$:
     - Compute $FIRST^k(\beta w)$ (concatenate $\beta$ and lookahead $w$, then take first $k$ symbols).
     - For each string $u \in FIRST^k(\beta w)$:
       - Add item $[B \to \bullet \gamma, u]$ to $CLOSURE(I)$ if not already present.
3. Repeat step 2 until no new items are added to $CLOSURE(I)$.

Steps to Compute $\mathbf{GOTO(I, X)}$ for item set $I$ and symbol $X$:
1. Initialize $J = \emptyset$.
2. For each item $[A \to \alpha \bullet X \beta, w]$ in $I$:
   - Add item $[A \to \alpha X \bullet \beta, w]$ to $J$.
3. Return $CLOSURE(J)$.

Steps to Construct the $\mathbf{LR(k) \text{ Automaton}}$:
1. Create the initial state $I_0 = CLOSURE(\{[S' \to \bullet S, \$^k]\})$ where $S'$ is the augmented start symbol.
2. Initialize the set of states $\mathcal{C} = \{I_0\}$ and a worklist $W = \{I_0\}$.
3. While $W \neq \emptyset$:
   - Remove a state $I$ from $W$.
   - For each symbol $X$ (terminal or non-terminal) such that $GOTO(I, X) \neq \emptyset$:
     - Let $J = GOTO(I, X)$.
     - If $J \notin \mathcal{C}$:
       - Add $J$ to $\mathcal{C}$ and $W$.
     - Add transition $I \xrightarrow{X} J$ to the automaton.
4. The resulting automaton defines the LR parsing table.




\section{Parsing Conflicts}

\textbf{Shift-Reduce Conflict}
A shift-reduce conflict occurs when the parser cannot decide whether to:
1) \textbf{Shift}: Move the next input symbol onto the stack
2) \textbf{Reduce}: Apply a production rule to reduce symbols on the stack


In terms of LR items, this happens when a state contains both:
1) A shift item: $[A \to \alpha \bullet a \beta, w]$ (expecting terminal $a$)
2) A reduce item: $[B \to \gamma \bullet, a]$ (ready to reduce with lookahead $a$)


The conflict manifests in the parsing table as both a shift action and a reduce action for entry $[state, a]$.
\textbf{Reduce-Reduce Conflict}
A reduce-reduce conflict occurs when the parser cannot decide which production to use for reduction. This happens when a state contains multiple reduce items with overlapping lookaheads:
1) $[A \to \alpha \bullet, w]$
2) $[B \to \beta \bullet, w]$


Both items indicate that reduction is possible with the same lookahead $w$, but the parser cannot determine which production rule to apply.

The conflict manifests in the parsing table as multiple reduce actions for the same entry $[state, w]$.

\textbf{Resolution}
These conflicts indicate that the grammar is not in the respective LR class (LR(k), LALR(k), or SLR(k)). Resolution strategies include:
1) Increasing the lookahead length $k$
2) Grammar transformation (left-factoring, eliminating ambiguity)
3) Using precedence and associativity rules
