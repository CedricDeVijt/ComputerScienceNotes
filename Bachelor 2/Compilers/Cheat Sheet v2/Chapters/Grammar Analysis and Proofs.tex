% 6. Grammar Analysis and Proofs
% Purpose: Systematic approach to classify grammars

% Preliminary Checks:

% Well-formed grammar verification
% Ambiguity detection
% Useless symbol removal


% Classification Procedures:

% Step-by-step algorithms for each parser type
% Conflict detection methods
% Testing different values of k


% Optimization Strategy: Using hierarchy to minimize work

\section{Grammar and language proof approach}
If you have to prove that the language generated by a grammar is LL(k) or LR(k), you can you can alter the grammar to make it easier to prove. The following steps can be used to systematically analyze and classify grammars:

\subsection{Preliminary checks}
\begin{itemize}
    \item \textbf{Well-formed grammar:} Check if the grammar is well-formed, i.e., it adheres to the rules of context-free grammars.
    \item \textbf{Ambiguity detection:} Identify if the grammar is ambiguous by checking for multiple parse trees for the same string.
    \item \textbf{Useless symbol removal:} Eliminate symbols that do not contribute to the language generated by the grammar.
\end{itemize}
