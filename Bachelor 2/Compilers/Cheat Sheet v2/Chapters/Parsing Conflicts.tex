\section{Parsing Conflicts}

\textbf{Shift-Reduce Conflict}

A shift-reduce conflict occurs when the parser cannot decide whether to:
\begin{enumerate}
\item \textbf{Shift}: Move the next input symbol onto the stack
\item \textbf{Reduce}: Apply a production rule to reduce symbols on the stack
\end{enumerate}

In terms of LR items, this happens when a state contains both:
\begin{enumerate}
\item A shift item: $[A \to \alpha \bullet a \beta, w]$ (expecting terminal $a$)
\item A reduce item: $[B \to \gamma \bullet, a]$ (ready to reduce with lookahead $a$)
\end{enumerate}

The conflict manifests in the parsing table as both a shift action and a reduce action for entry $[state, a]$.


\textbf{Reduce-Reduce Conflict}

A reduce-reduce conflict occurs when the parser cannot decide which production to use for reduction. This happens when a state contains multiple reduce items with overlapping lookaheads:
\begin{enumerate}
\item $[A \to \alpha \bullet, w]$
\item $[B \to \beta \bullet, w]$
\end{enumerate}

Both items indicate that reduction is possible with the same lookahead $w$, but the parser cannot determine which production rule to apply.

The conflict manifests in the parsing table as multiple reduce actions for the same entry $[state, w]$.


\textbf{Resolution}

These conflicts indicate that the grammar is not in the respective LR class (LR(k), LALR(k), or SLR(k)). Resolution strategies include:
\begin{enumerate}
\item Increasing the lookahead length $k$
\item Grammar transformation (left-factoring, eliminating ambiguity)
\item Using precedence and associativity rules
\end{enumerate}
