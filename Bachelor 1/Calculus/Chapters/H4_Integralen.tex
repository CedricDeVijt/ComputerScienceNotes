\section{H4 Integralen}

\subsection{Basics}

$\displaystyle\int x^n dx = \dfrac{x^{n+1}}{n+1}+C$\\
$\displaystyle\int f(x)+g(x)dx = \displaystyle\int f(x)dx + \displaystyle\int g(x)dx$\\
$\displaystyle\int \lambda f(x)dx = \lambda \displaystyle\int f(x)dx$\\
$\displaystyle\int f(x)g'(x)dx = f(x)g(x)-\displaystyle\int f'(x)g(x)dx$\\
$\displaystyle\int \dfrac{f'(x)}{f(x)}dx = \ln |f(x)| + C$\\

\subsection{Goniometrische functies}

$\displaystyle\int \sin x dx = -\cos x + C$\\
$\displaystyle\int \cos x dx = \sin x + C$\\
$\displaystyle\int \dfrac{1}{\cos^2 x}dx = \tan x + C$\\
$\displaystyle\int \dfrac{-1}{\sin^2 x}dx = \cot x + C$\\

\subsection{Cyclometrische functies}

$\displaystyle\int \dfrac{1}{\sqrt{1-x^2}}dx = \operatorname{Bgsin} x + C$\\
$\displaystyle\int \dfrac{1}{\sqrt{a^2-x^2}}dx = \operatorname{Bgsin} \dfrac{x}{a} + C$\\
$\displaystyle\int \dfrac{1}{x^2+1}dx = \operatorname{Bgtan} x + C$\\
$\displaystyle\int \dfrac{1}{\sqrt{x^2+a}}dx = \ln |x+\sqrt{x^2+a}| + C$\\

\subsection{Hyperbolische functies}

$\displaystyle\int \cosh x dx = \sinh x + C$\\
$\displaystyle\int \sinh x dx = \cosh x + C$\\
$\displaystyle\int \dfrac{1}{\cosh^2 x}dx = \tanh x + C$\\
$\displaystyle\int \dfrac{1}{\sinh^2 x}dx = -\coth x + C$\\

\subsection{Exponentiële functies}

$\displaystyle\int a^x dx = \dfrac{a^x}{\ln a} + C$\\
$\displaystyle\int e^x dx = e^x + C$\\
$\displaystyle\int \dfrac{1}{x}dx = \ln |x| + C$\\