\section{Counting}

\subsection{Basic Techniques}

\subsubsection*{Sum Rule}
\textbf{Property 4.1 (Sum Rule)}: If there are $n(A)$ ways to perform $A$ and $n(B)$ ways to perform $B$, then the total number of ways to perform $A$ \textbf{or} $B$ is $n(A) + n(B)$. This extends to multiple events:
\begin{itemize}
    \item $n(A) + n(B) + n(C)$ ways to perform $A$, $B$, or $C$, etc.
\end{itemize}

\subsubsection*{Product Rule}
\textbf{Property 4.2 (Product Rule)}: If there are $n(A)$ ways to perform $A$ and $n(B)$ ways to perform $B$, and these are independent, the total number of ways to perform $A$ \textbf{and} $B$ is $n(A) \cdot n(B)$. This generalizes as:
\begin{itemize}
    \item $n(A) \cdot n(B) \cdot n(C)$ ways to perform $A$, $B$, and $C$, etc.
\end{itemize}

\subsubsection*{Division Rule}
\textbf{Property 4.3 (Division Rule)}: If there is a $k$-to-1 correspondence between objects of type $A$ and type $B$, and there are $n(A)$ objects of type $A$, then there are $n(B) = \frac{n(A)}{k}$ objects of type $B$.

\subsection{Inclusion-Exclusion Principle}

\textbf{Example 4.4}: In a class, 20 students have a driver’s license, 16 have a bus pass, and 7 have both. How many students have a bus pass or a driver’s license?
\begin{itemize}
    \item Add 20 and 16, then subtract the overlap: $20 + 16 - 7 = 29$.
\end{itemize}

\textbf{Property 4.5 (Inclusion-Exclusion for 2 Sets)}:
\begin{align*}
    |A \cup B| = |A| + |B| - |A \cap B|
\end{align*}

\textbf{Property 4.6 (Inclusion-Exclusion for 3 Sets)}:
\begin{align*}
    |A \cup B \cup C| = |A| + |B| + |C| - |A \cap B| - |A \cap C| - |B \cap C| + |A \cap B \cap C|
\end{align*}

For $n$ sets, a generalization is:
\begin{align*}
    \left| \bigcup_{i=1}^n A_i \right| = \sum_{i=1}^n |A_i| - \sum_{1 \leq i < j \leq n} |A_i \cap A_j| + \dots + (-1)^{n-1} \left| \bigcap_{i=1}^n A_i \right|.
\end{align*}

\subsection{Decision Trees}

\textbf{Example 4.8}: A staircase has 4 steps. How many ways can you climb it, taking 1, 2, 3, or 4 steps at a time?
\begin{itemize}
    \item Draw a tree for possibilities. Each complete path to the top corresponds to a way. Total: 8 ways.
\end{itemize}

\subsection{Permutations and Combinations}

\subsubsection*{Variations}
\textbf{Definition 4.9}: A variation of $k$ objects from $n$ is an ordered selection of $k$ objects from $n$, without repetition. If $n = k$, it’s a permutation.
\begin{itemize}
    \item Number of variations:
    \begin{align*}
        V_k^n = \frac{n!}{(n-k)!}
    \end{align*}
    \item Permutations: $P_n = n!$
\end{itemize}

\subsubsection*{Combinations}
\textbf{Definition 4.11}: A combination of $k$ objects from $n$ is a selection of $k$ objects without regard to order, without repetition.
\begin{itemize}
    \item Number of combinations:
    \begin{align*}
        C_k^n = \binom{n}{k} = \frac{n!}{k!(n-k)!}
    \end{align*}
\end{itemize}

\subsubsection*{Generalizations}
\textbf{Definition 4.12}: Allowing repetitions:
\begin{align*}
    D_k^n = C_k^n = \binom{n+k-1}{k}
\end{align*}
