\section{Generating Functions}

\subsection*{Definitions and Concepts}

\textbf{Generating Function for a Sequence:}
Given a sequence $a_0, a_1, a_2, \dots$, the generating function $G(x)$ is defined as:
\[
G(x) = a_0 + a_1x + a_2x^2 + \dots = \sum_{n=0}^\infty a_nx^n.
\]

\textbf{Formal Power Series:}
A formal power series is an expression of the form:
\[
a_0 + a_1x + a_2x^2 + a_3x^3 + \dots,
\]
where coefficients $a_n$ are given but the series may not converge.

\subsection*{Useful Generating Functions}

\begin{itemize}
    \item \textbf{Geometric Series:}
    \[ \sum_{n=0}^\infty x^n = \frac{1}{1-x}, \quad |x| < 1. \]

    \item \textbf{Generalized Geometric Series:}
    \[ \sum_{n=0}^\infty c^nx^n = \frac{1}{1-cx}, \quad |cx| < 1. \]

    \item \textbf{Powers of $(1-x)^{-m}$:}
    \[ \frac{1}{(1-x)^m} = \sum_{n=0}^\infty \binom{n+m-1}{m-1}x^n, \quad |x| < 1. \]

    \item \textbf{Derivative Formulas:}
    \[ \frac{x}{(1-x)^2} = \sum_{n=1}^\infty nx^n, \]
    \[ \frac{1}{(1-x)^2} = \sum_{n=0}^\infty (n+1)x^n. \]

    \item \textbf{Exponential Generating Function:}
    \[ \sum_{n=0}^\infty \frac{x^n}{n!} = e^x. \]

    \item \textbf{Alternate Series:}
    \[ \sum_{n=0}^\infty (-1)^n x^n = \frac{1}{1+x}. \]
\end{itemize}

\subsection*{Examples}

\textbf{Example 1: Fruit Selection}
\begin{align*}
    &\text{Given a fruit basket with 2 apples, 1 pear, 1 plum, and 1 banana, the generating function is:} \\
    &\quad G(x) = (1+x+x^2)(1+x)(1+x)(1+x).
\end{align*}
The coefficient of $x^2$ in $G(x)$ gives the number of ways to choose 2 fruits.

\textbf{Example 2: Pastries}
\begin{align*}
    &\text{For 3 cheese pastries, 2 apricot pastries, and 4 strawberry pastries, the generating function is:} \\
    &\quad G(x) = (1+x+x^2+x^3)(1+x+x^2)(1+x+x^2+x^3+x^4).
\end{align*}

\subsection*{Operations on Generating Functions}

\textbf{Addition:} If $A(x)$ and $B(x)$ are generating functions, their sum corresponds to termwise addition of coefficients:
\[ (a_0 + a_1x + \dots) + (b_0 + b_1x + \dots) = (a_0 + b_0) + (a_1 + b_1)x + \dots. \]

\textbf{Multiplication:} The product of generating functions corresponds to convolution of coefficients:
\[ A(x)B(x) = \sum_{n=0}^\infty \left( \sum_{k=0}^n a_k b_{n-k} \right)x^n. \]

\subsection*{Inverse Generating Functions}

A generating function $S(x)$ with $S(0) \neq 0$ has an inverse $T(x)$ such that:
\[ S(x)T(x) = 1. \]

Example:
\[ S(x) = 1+2x+3x^2+\dots \implies T(x) = 1-2x+x^2. \]

\subsection*{Applications}

\textbf{Solving Recurrence Relations:}
Generating functions can transform recurrence relations into algebraic equations. For example:
\begin{align*}
    h_n &= 2h_{n-1} + 1, \quad h_0 = 0. \\
    \text{Generating function: } H(x) &= \frac{x}{(1-x)(1-2x)}.
\end{align*}

\textbf{Finding Closed Forms:}
For a recurrence $s_n = -s_{n-1} + 6s_{n-2}$ with $s_0 = 1, s_1 = 1$, we get:
\[ S(x) = \frac{1+2x}{(1+3x)(1-2x)}. \]