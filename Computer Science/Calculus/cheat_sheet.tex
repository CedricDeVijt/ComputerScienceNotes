\documentclass[10pt,landscape]{article}
\usepackage{amssymb,amsmath,amsthm,amsfonts}
\usepackage{multicol,multirow}
\usepackage{calc}
\usepackage{ifthen}
\usepackage[landscape,a4paper,top=1cm,left=1cm,right=1cm,bottom=1cm]{geometry}
\usepackage[colorlinks=true,citecolor=blue,linkcolor=blue]{hyperref}

\pagestyle{empty}
\makeatletter
\renewcommand{\section}{\@startsection{section}{1}{0mm}%
                                {-1ex plus -.5ex minus -.2ex}%
                                {0.5ex plus .2ex}%
                                {\normalfont\large\bfseries}}
\renewcommand{\subsection}{\@startsection{subsection}{2}{0mm}%
                                {-1ex plus -.5ex minus -.2ex}%
                                {0.5ex plus .2ex}%
                                {\normalfont\normalsize\bfseries}}
\renewcommand{\subsubsection}{\@startsection{subsubsection}{3}{0mm}%
                                {-1ex plus -.5ex minus -.2ex}%
                                {1ex plus .2ex}%
                                {\normalfont\small\bfseries}}
\makeatother
\setcounter{secnumdepth}{0}
\setlength{\parindent}{0pt}
\setlength{\parskip}{0pt plus 0.5ex}

% -----------------------------------------------------------------------

\title{Calculus}

\begin{document}

\raggedright
\footnotesize

\begin{center}
     \Large{\textbf{Calculus}} \\
\end{center}
\begin{multicols}{3}
\setlength{\premulticols}{1pt}
\setlength{\postmulticols}{1pt}
\setlength{\multicolsep}{1pt}
\setlength{\columnsep}{2pt}

\section{H1 Getallenverzameling}
$r=\sqrt{(x^2+y^2)}$\\
$\theta=\arctan \left(\dfrac{y}{x}\right)$\\
$z=r\operatorname{cis}\alpha = r(\cos \alpha + i\sin \alpha)$\\
$(r\operatorname{cis}\alpha)^n=r^n\operatorname{cis}(n\alpha)$\\
$z_k=\sqrt[n]{r} \operatorname{cis} \left(\alpha + \frac{2k\pi}{n} \right)$


\section{H2 Limieten}

\subsection{Exponentiële en logaritmische functies}
$\lim\limits_{x\rightarrow \infty}\left(1+\dfrac{1}{x}\right)^x=e$
$\lim\limits_{x\rightarrow \infty}\left(1+\dfrac{k}{x}\right)^x=e^k$

\subsection{Bijzondere goniometrische limieten}
$\lim\limits_{x\rightarrow0}\dfrac{\sin (x)}{x} = 1$
$\lim\limits_{x\rightarrow 0}\dfrac{\tan (x)}{x}=1$

\vfill\null
\columnbreak

\section{H3 Afgeleiden}

\subsection{Basics}
$D(x^n)=nx^{n-1}dx$\\
$D(f(x)+g(x))= D(f(x))+ D(g(x))$\\
$D(\lambda f(x))= \lambda D(f(x))$\\
$d(f \cdot g)(x) = f(x)g'(x)+f'(x)g(x)$\\
$d\left( \dfrac{f(x)}{g(x)}\right) = \dfrac{g(x)f'(x)-g'(x)f(x)}{(g(x))^2}$\\

\subsection{Goniometrische functies}

$D(\sin x)=\cos x$\\
$D(\cos x)=-\sin x$\\
$D(\tan x)=\dfrac{1}{\cos^2 x}$\\
$D(\cot x)=\dfrac{-1}{\sin^2 x}$\\
$D(\sec x)=\dfrac{\sin x}{\cos^2 x}$\\
$D(\csc x)=\dfrac{-\cos x}{\sin^2 x}$\\

\subsection{Cyclometrische functies}

$D(\operatorname{Bgsin} x)=\dfrac{1}{\sqrt{1-x^2}}$\\
$D(\operatorname{Bgcos} x)=\dfrac{-1}{\sqrt{1-x^2}}$\\
$D(\operatorname{Bgtan} x)=\dfrac{1}{1+x^2}$\\
$D(\operatorname{Bgcot} x)=\dfrac{-1}{1+x^2}$\\

\subsection{Hyperbolische functies}

$D(\sinh x)=\cosh x$\\
$D(\cosh x)=\sinh x$\\
$D(\tanh x)=\dfrac{1}{\cosh^2 x}$\\
$D(\coth x)=\dfrac{-1}{\sinh^2 x}$\\

\subsection{Exponentiële functies}

$D(a^x)=a^x \ln a$\\
$D(e^x)=e^x$\\
$D(\ln x)=\dfrac{1}{x}$\\
$D(\log_a x)=\dfrac{1}{x \ln a}$\\

\subsection{Kettingregel}

$D(f(g(x)))=f'(g(x))g'(x)$\\

\subsection{Machtsregel}
$D(f(x)^{g(x)})=g(x)f(x)^{g(x)-1}f'(x)+f(x)^{g(x)}g'(x)\ln f(x)$\\

\vfill\null
\columnbreak

\section{H4 Integralen}

\subsection{Basics}

$\displaystyle\int x^n dx = \dfrac{x^{n+1}}{n+1}+C$\\
$\displaystyle\int f(x)+g(x)dx = \displaystyle\int f(x)dx + \displaystyle\int g(x)dx$\\
$\displaystyle\int \lambda f(x)dx = \lambda \displaystyle\int f(x)dx$\\
$\displaystyle\int f(x)g'(x)dx = f(x)g(x)-\displaystyle\int f'(x)g(x)dx$\\
$\displaystyle\int \dfrac{f'(x)}{f(x)}dx = \ln |f(x)| + C$\\

\subsection{Goniometrische functies}

$\displaystyle\int \sin x dx = -\cos x + C$\\
$\displaystyle\int \cos x dx = \sin x + C$\\
$\displaystyle\int \dfrac{1}{\cos^2 x}dx = \tan x + C$\\
$\displaystyle\int \dfrac{-1}{\sin^2 x}dx = \cot x + C$\\

\subsection{Cyclometrische functies}

$\displaystyle\int \dfrac{1}{\sqrt{1-x^2}}dx = \operatorname{Bgsin} x + C$\\
$\displaystyle\int \dfrac{1}{\sqrt{a^2-x^2}}dx = \operatorname{Bgsin} \dfrac{x}{a} + C$\\
$\displaystyle\int \dfrac{1}{x^2+1}dx = \operatorname{Bgtan} x + C$\\
$\displaystyle\int \dfrac{1}{\sqrt{x^2+a}}dx = \ln |x+\sqrt{x^2+a}| + C$\\

\subsection{Hyperbolische functies}

$\displaystyle\int \cosh x dx = \sinh x + C$\\
$\displaystyle\int \sinh x dx = \cosh x + C$\\
$\displaystyle\int \dfrac{1}{\cosh^2 x}dx = \tanh x + C$\\
$\displaystyle\int \dfrac{1}{\sinh^2 x}dx = -\coth x + C$\\

\subsection{Exponentiële functies}

$\displaystyle\int a^x dx = \dfrac{a^x}{\ln a} + C$\\
$\displaystyle\int e^x dx = e^x + C$\\
$\displaystyle\int \dfrac{1}{x}dx = \ln |x| + C$\\

\vfill\null
\columnbreak

\vbox{
\section{H5 Bepaalde integralen}

\subsection{Oppervlakte}

\begin{tabular}{|l|l|}
    \hline
    Cartesisch & \( S=\displaystyle\int_a^b \|f(x)\|dx \) \\
    \hline
    Parameter & \( S=\displaystyle\int_a^b \|g(t)\|f'(t)dt \) \\
    \hline
    Pool & \( S=\displaystyle\int_{\alpha}^{\beta} (r(\theta))^2d\theta \) \\
    \hline
\end{tabular}

\subsection{Omwentelingsvolume}

\begin{tabular}{|l|l|}
    \hline
    Cartesisch & \( V=\pi \displaystyle\int_a^b (f(x))^2dx \) \\
    \hline
    Parameter & \( V=\pi \displaystyle\int_a^b (g(t))^2f'(t)dt \) \\
    \hline
    Pool & \\
    \hline
\end{tabular}

\subsection{Booglengte}

\begin{tabular}{|l|l|}
    \hline
    Cartesisch & \( L=\displaystyle\int_a^b \sqrt{1+(f'(x))^2}dx \) \\
    \hline
    Parameter & \( L=\displaystyle\int_a^b \sqrt{(f'(t))^2+(g'(t))^2}dt \) \\
    \hline
    Pool & \( L=\displaystyle\int_{\alpha}^{\beta} \sqrt{(r(\theta))^2+(r'(\theta))^2}d\theta \) \\
    \hline
\end{tabular}

\subsection{Complanatie}

\begin{tabular}{|l|l|}
    \hline
    Cartesisch & \( C=2\pi \displaystyle\int_a^b \|f(x)\| \sqrt{1+(f'(x))^2}dx \) \\
    \hline
    Parameter & \( C=2\pi \displaystyle\int_a^b \|g(t)\| \sqrt{(f'(t))^2+(g'(t))^2}dt \) \\
    \hline
    Pool & \\
    \hline
\end{tabular}
}

\end{multicols}
\end{document}
